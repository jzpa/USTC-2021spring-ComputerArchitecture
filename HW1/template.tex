%!TEX program = xelatex
\documentclass[12pt,a4paper,utf8]{ctexart}
\usepackage{graphicx}
\usepackage{amsmath}
\usepackage{amssymb}
\usepackage{subfig}
\usepackage{cite}
\usepackage[ntheorem]{empheq}
\usepackage{enumitem}
\usepackage{fullpage}
\usepackage{cleveref}
\usepackage{cellspace}
\usepackage{listings}
\usepackage{color}
\definecolor{gray}{rgb}{0.5,0.5,0.5}
\definecolor{dkgreen}{rgb}{.068,.578,.068}
\definecolor{dkpurple}{rgb}{.320,.064,.680}

% set Matlab styles
\lstset{
   language=Matlab,
   keywords={break,case,catch,continue,else,elseif,end,for,function,
      global,if,otherwise,persistent,return,switch,try,while},
   basicstyle=\ttfamily,
   keywordstyle=\color{blue}\bfseries,
   commentstyle=\color{dkgreen},
   stringstyle=\color{dkpurple},
   backgroundcolor=\color{white},
   tabsize=4,
   showspaces=false,
   showstringspaces=false
}

\begin{document}

\CJKfamily{zhkai}       


\begin{center}
\textbf{作业一}\\
\textbf{林成渊 ~~~~~ PB18051113 ~~~~~ \zhtoday}\\
\end{center}
\textit{}
\vspace{\baselineskip}

\begin{enumerate}
\item[EX1] %\textbf{重心插值公式(barycentric interpolation formula)}  
某台计算机执行标准测试程序,程序中指令类型、出现频率、需要时钟周期数如下:
\begin{table}[h!]
	\begin{center}
		%\caption{Your first table.}
		\begin{tabular}{|c|c|c|}
			\hline
			指令类型 & 指令出现频率 & 需要时钟周期数 \\
			\hline
			R & 30\% & 2 \\
			\hline
			I & 25\% & 3 \\
			\hline
			S & 20\% & 2 \\
			\hline
			U & 15\% & 4 \\
			\hline
			B & 5\% & 4 \\
			\hline
			J & 5\% & 2 \\
			\hline
		\end{tabular}
	\end{center}
\end{table}

\textbf{1.1 计算CPI}

解:由题意列式计算有
\begin{equation}
	\begin{aligned}
		CPI & = 30\% \times 2 + 25\% \times 3 + 20\%\times 2 + 15\% \times 4 + 5\% \times 4 + 5\% \times 2 \\
			& = 2.65
	\end{aligned}
	\nonumber
\end{equation}

	
\textbf{1.2 有以下两种方案进行优化:}
\begin{itemize}
	\item \textbf{A.}整体时钟周期时间缩短到原本的0.9
	\item \textbf{B.}B类型和U类型指令需要的时钟周期数减少1
\end{itemize}
试比较这两种方案

解:分别计算两种方案的加速比如下
\begin{equation}
	\begin{aligned}
		\mbox{Case A:} \quad \mu_A 	&= 1 / 0.9 \\
								&= \frac{10}{9} \\
								&\approx 111.11\% \\
		\mbox{Case B:} \quad \mu_B	&= t_{old} / t_{new} \\
									&= \bigg( 30\% \times 2 + 25\% \times 3 + 20\%\times 2 + 15\% \times 4 + 5\% \times 4 + 5\% \times 2 \bigg)  \bigg/ \\ 
									&\qquad \bigg( 30\% \times 2 + 25\% \times 3 + 20\%\times 2 + 15\% \times 3 + 5\% \times 3 + 5\% \times 2\bigg) \\
								&= \frac{53}{49} \\
								&\approx 108.16\%
	\end{aligned}
	\nonumber
\end{equation}
可见加速比$ \mu_A > \mu_B $,选择方案A更佳

\item[EX2] %\textbf{重心插值公式(barycentric interpolation formula)}  
我们通过添加高性能硬件模块来提升机器的性能,当计算通过高性能模块进行加速时,其速度是正常运行的20倍,将通过高性能模块进行加速的运算花费的时间百分比记为$ \alpha $(加速后所测得执行时间百分比)

\textbf{2.1 $\alpha $达到多少时,运算总体加速比达到$3$}

解:根据题意列式得
\begin{equation}
	\begin{aligned}
		\mbox{加速比} &= t_{old} / t_{new} \\
						&= (20 \times \alpha + (1 - \alpha)) / 1 \\
						&= 3
	\end{aligned}
	\nonumber
\end{equation}
解得$\alpha = 2/19$

\textbf{2.2 在整体加速比为$3$的情况下,被加速的计算在原执行时间中占比例为多少}

解:引用2.1题数据列式得
\begin{equation}
	\begin{aligned}
		\mbox{比例} &= (20 \times \alpha ) / (20 \times \alpha + (1 - \alpha)) \\
					&= \frac{40}{57}
	\end{aligned}
	\nonumber
\end{equation}

\textbf{2.3 $\alpha $达到多少时,运算整体加速比能达到此加速方式最大加速比的一半}

解:关于“此加速方式的最大加速比”存在两种理解,如看作加速部分指令原来的比例不变而加速的倍率可以提至无限,根据题意列式得
\begin{equation}
	\begin{aligned}
		\mbox{运算整体加速比}		&= t_{old} / t_{new} \\
								&= (20 \times \alpha + (1 - \alpha)) / 1 \\
								&= \mbox{此加速方式最大加速比} \bigg / 2 \\
								&= t_{old} / t_{ideal} \\
								&= (19\times \alpha + 1) / (1 - \alpha )
	\end{aligned}
	\nonumber
\end{equation}
解得$\alpha = 1/2$

但是若看作加速的倍率不变而加速部分指令原来的比例可变,根据题意列式得
\begin{equation}
	\begin{aligned}
		\mbox{运算整体加速比}		&= t_{old} / t_{new} \\
		&= (20 \times \alpha + (1 - \alpha)) / 1 \\
		&= \mbox{此加速方式最大加速比} \bigg / 2 \\
		&= 20 / 2 \\
		&= 10
	\end{aligned}
	\nonumber
\end{equation}
解得$\alpha = 9/19$

\item[EX3] %\textbf{重心插值公式(barycentric interpolation formula)}  
我们为一种实时应用设计系统,这种应用要求必须在指定期限之前完成,提前完成计算没有收益。我们发现,在最糟糕的情况下,这一系统执行必需代码的速度是最低要求速度的两倍

\textbf{3.1 如果以当前速度执行计算,并在完成任务后关闭系统,可节省多少能量}

解:由于当前速度是最低要求速度的两倍,期限之前完成任务仅需一半时间,另外一半时间内系统处于关闭状态,故节省了$50\%$的能量

\textbf{3.2 如果将电压和频率设置为现在的一半,可以节省多少能量}

解:其余条件不变的情况下,功率$W\propto \mbox{电压}^2 \times \mbox{频率}$,所以节省了$1-1/2^3 = 7/8 = 87.5\%$的能量


\end{enumerate}




\end{document}
